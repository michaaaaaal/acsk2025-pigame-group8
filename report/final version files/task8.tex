\paragraph{}
During our work with the newsvendor problem, we came across both strengths and weaknesses of the methods introduced during the course, that are worth discussing. Using parametric models, like normal and log-normal distributions, was a convenient solution that let us determine optimal order quantities with relative ease. However, these models heavily rely on distributional assumptions that may not be fully available in reality, due to the distributional uncertainty we are often met with when analyzing real-world datasets. For instance, stores such as Station B, which experience multimodal or very skewed demand, made us question the suitability of classic distribution forms. On the other hand, non-parametric approaches that we introduced through bootstrap methods offered much more flexibility, but at lower sample sizes they became increasingly less reliable, and their estimations were inconsistent. 


Our modeling approach was built on a number of necessary assumptions. We presumed that demand was independent and identically distributed (i.i.d.) over time, and that all values related to cost such as those of production, clearance, and shipping could be expressed as stable average values. These assumptions enabled us to perform a manageable analysis that aligned with theoretical concepts introduced and discussed during the lectures, however they inevitably gloss over the intricate details and difficulties that retail operations face in reality.

For example, demand often varies due to external factors like holidays, economic conditions or even weather changes, none of which we directly took into account and modeled. Similarly, the fixed price-demand relationship we used in Task 7, despite being based on our internal research, operates on the assumption that consumer sensitivity is uniform across different locations and times, which might not exactly be the case in reality.

Despite these limitations, applying both theoretical and empirical models significantly helped our understanding of stochastic inventory management. The difference between parametric and nonparametric methods, which initially could have seemed abstract, became much more understandable as we applied them to real data. We noticed that parametric models work well when the assumed distribution matches the provided data, like in the case of \emph{Main Street A}, but can mislead when the fit is not as clear.

In contrast, nonparametric methods proved to be more robust under uncertain assumptions, although they required careful management of sampling variability. Task 7, in particular, improved our understanding of the relationship between customer behavior and operational decisions. We simulating the impact of price increases on demand, which showed how it influences optimal ordering. We showcased the practical benefit of merging profit modeling with behavioral economics.

Looking ahead, there is plenty of room for improvement. Future research could integrate time-series models to take into account autocorrelation temporal trends in demand, especially for stores that experience seasonal changes. Additionally, incorporating external factors like local events or competitor pricing into our demand model could improve future forecast accuracy.

From a methodological standpoint, considering robust optimization or Bayesian methods might provide alternatives that better address the issues of parameter uncertainty or model misspecification. Lastly, shifting our focus from aggregate demand to considering customer-level differences in price sensitivity would allow for more refined inventory and pricing optimizations, ultimately boosting both customer satisfaction and profitability of the operations.

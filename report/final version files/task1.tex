\noindent The Newsvendor Problem (NVP) addresses a classic model used in inventory management that helps determine how much stock a decision-maker, facing uncertain demand, should order. It is most relevant for perishable products or seasonal goods because their value decreases over time, meaning that an overstocking results in financial loss. The purpose of the model is to determine the optimal order quantity that takes the risk of having too little inventory and the risk of ordering too much and balances them out to maximize expected profit.

The NVP provides a solid foundation for solving real-world decision-making problems when faced with uncertainty. In practice, companies often face the challenge of having to interpret varying historical demand data and derive feasible stocking decisions from them. For example, a clothing store might rely on past sales to forecast demand for seasonal items such as winter jackets. The model allows them to use these forecasts to determine an optimal order quantity that balances the risk of overstocking with the risk of missing potential sales. This trade-off also applies to services. Mobile service providers such as Vodafone face similar trade-offs when designing their bundles. An order quantity that is too high can lead to unused resources, while one that is too low risks customer dissatisfaction. What makes the NVP especially useful here is its mathematical convenience. When the demand distribution is known, the optimal order quantity corresponds to a specific quantile of that distribution, which can be computed using Python. The ability to go from probabilistic modeling to precise numerical solutions is what makes the NVP a powerful tool in data science.

However, the NVP has its limitations and challenges. A big challenge in real-life applications of the Newsvendor Problem is the difficulty in demand estimation. Accurate modeling of demand distribution is often hard due to limited data, seasonal trends, or other various outside factors. For instance, the COVID-19 pandemic massively affected consumer behavior, making historical data much less reliable. Moreover, real costs of under- or over-stocking are not always fully shown by the cost and price parameters. To tackle this, we introduce methods such as parametric estimation (for example fitting normal or Poisson distributions to demand data) and nonparametric, quantile-based estimators. These allow decision-makers to do their job without having to rely solely on strict assumptions. Machine learning methods can improve forecasting of demand and simulations like the Monte Carlo one methods can help evaluate how different order quantities perform in different scenarios. 

These methods extend beyond inventory settings. For example, in healthcare logistics, quantile-based models are used to manage uncertain demand for critical resources such as hospital beds or blood products. \cite{shen2024} use a data-driven newsvendor framework to balance elective and emergency patient admissions under an uncertain hospital capacity. Their approach avoids assuming a known demand distribution and instead uses the historical data to estimate the optimal service level. This shows how quantile-based decision rules remain effective even when faced with unstable, high-stakes demand.
Similarly, \cite{patra2021} use a two-period newsvendor model in disaster relief logistics. They optimize how much aid to station before a disaster and how much to send afterward, using Bayesian updates to reflect real-time information. This expansion shows how newsvendor-based reasoning applies in emergency situations, where the cost of over- or under-supplying is both financial and human.
In this project, we will explore its relevance in a more everyday setting: a bakery chain from Vilnius. This project builds on the classical NVP and applies its logic and methods to a practical case study involving the day-to-day operations of the business. The business faces a previously covered tradeoff: bake too little and miss out on potential profit, or bake too much and either create waste or lose money on clearance sales. There are also additional factors such as demand changes for different days of the week, store-specific characteristics, and external factors such as the COVID-19 pandemic. Our goal is to estimate optimal production quantities that maximize expected profit using historical demand data.

When working on the Vilnius bakery problem, we will use both parametric and nonparametric estimation methods introduced in the course. In Task 2, we will revise the classical Newsvendor profit function to take into account the clearance price and shipping cost, making the model more applicable in the context of the bakery. We will analyze how increasing the shipping cost \( c_S \) will affect the optimal order quantity and the expected profit.

In Task 3, we will perform a Monte Carlo simulation study to compare the performance of the parametric and nonparametric estimators across different service levels and sample sizes. We will focus on the Root Mean Squared Error (RMSE), Profit Loss Ratio (PLR), and Service Level (SL). The goal is to confirm the assumption that, under correctly fitted distributions, the parametric method outperforms the nonparametric one in both accuracy and profit outcomes. However, we observe that the nonparametric estimator will remain useful when the true demand distribution is uncertain.

In Task 4, we will test how the modeling error of the demand distribution impacts the estimate of order quantity. We will generate demand data from non-normal distributions (Poisson and Exponential) and then fit a normal distribution to each sample. For every value $\tau$ we had, we estimate the value of $\hat{Q}$ using the fitted normal distribution and compare it to the true quantile $Q^*$ of the actual distribution by calculating their absolute error. To avoid any uncertainty caused by noise or randomness, we will run the simulation 1000 times for different values of $\tau$ and various sample sizes to evaluate how the misspecification error differs across quantiles and distributions when assuming a normal model incorrectly.

So far, our findings show that extending the classical Newsvendor model to include the shipping costs and clearance prices results in more realistic stocking recommendations. When the model is correctly specified, the parametric estimator outperforms the non-parametric one both in RMSE and PLR, but the latter is more robust when facing uncertainty or model misspecification. These findings suggest a hybrid approach: Use parametric models when the distribution is known and nonparametric when the data and knowledge are uncertain.

The following sections build on this. Section 2.2 elaborates on the theoretical aspects of the Newsvendor Problem and its methods. Section 2.3 applies these models to a real-world problem and evaluates the Vilnius bakery dataset. Finally, Section 2.4 reflects on the strengths and limitations of the approaches we used and discusses potential improvements and further research.
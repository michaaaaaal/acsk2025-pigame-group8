The newsvendor model is applicable in this case as the bakery faces uncertain demand, while selling same-day perishable goods. This creates a trade-off between the risks of lost sales (understocking) and spoilage (overstocking). Given demand variability and unknown true distribution, optimal order quantity must be estimated.

One method assumes a parametric distribution \( f(x; \theta) \), e.g., normal, with parameters estimated from historical data. Bootstrap samples \( X^* \sim f(x; \theta) \) are then generated.

Alternatively, the nonparametric approach resamples directly from the empirical distribution function (EDF), defined as:
\[
F(y) = \frac{1}{n} \sum_{i=1}^{n} H(y - y_i)
\]
where \( H(u) \) is the unit step function (1 if \( u \geq 0 \), 0 otherwise).

\paragraph{Root-Mean-Square Error (RMSE)} 
\[
\text{RMSE} = \sqrt{ \frac{1}{M} \sum_{i=1}^{n} (P_i - O_i)^2 }
\]
This measures the average magnitude of prediction error, where \( P_i \) are predicted and \( O_i \) observed values (see \cite{willmott2005} for discussion on the advantages of RMSE over MAE)

\paragraph{Empirical Price Loss Ratio (PLR)} 
\[
\text{PLR}_n^k(\tau) = \frac{1}{M} \sum_{j=1}^M \left| \frac{R(Q^*; \tau) - R(\hat{Q}_{n,j}^k(\tau); \tau)}{R(Q^*; \tau)} \right|
\]
PLR compares profit between the estimated order and optimal order, normalized by the optimal expected profit.

\paragraph{Empirical Service Level (SL)} 
\[
\text{SL}_n^k(\tau) = \frac{1}{M} \sum_{j=1}^M I\left\{ \hat{Q}_{n,j}^k(\tau) \geq Y_j \right\}
\]
This measures the percentage of simulations in which the estimated order quantity covers the actual demand.

\begin{table}[H]
\centering
\caption{RMSE$^P_n(\tau)$}
\label{tab:rmse_p}
\renewcommand{\arraystretch}{1.2}
\begin{tabular}{|c|c|c|c|c|c|c|c|c|c|c|}
\hline
$n \backslash \tau$ & 0.01 & 0.05 & 0.1 & 0.3 & 0.5 & 0.7 & 0.9 & 0.95 & 0.99 \\
\hline
10  & 6.212218 & 4.899544 & 4.472625 & 3.344246 & 3.24019 & 3.36837 & 4.37373 & 4.882186 & 6.219007 \\
50  & 2.76907  & 2.18982  & 1.850608 & 1.481755 & 1.465837 & 1.494265 & 1.899898 & 2.154807 & 2.736072 \\
100 & 1.905263 & 1.537951 & 1.336774 & 1.080217 & 1.014662 & 1.083056 & 1.33693  & 1.563087 & 1.881672 \\
200 & 1.354468 & 1.044762 & 0.959343 & 0.771061 & 0.692413 & 0.741837 & 0.947804 & 1.057911 & 1.368015 \\
\hline
\end{tabular}
\end{table}

\begin{table}[H]
\centering
\caption{RMSE$^{NP}_n(\tau)$}
\label{tab:rmse_np}
\renewcommand{\arraystretch}{1.2}
\begin{tabular}{|c|c|c|c|c|c|c|c|c|c|c|}
\hline
$n \backslash \tau$ & 0.01 & 0.05 & 0.1 & 0.3 & 0.5 & 0.7 & 0.9 & 0.95 & 0.99 \\
\hline
10  & 9.71554  & 5.972408 & 6.361219 & 4.463283 & 4.097794 & 4.205848 & 5.503998 & 5.828964 & 9.744622 \\
50  & 4.806933 & 2.854802 & 2.551798 & 1.861882 & 1.826107 & 1.856143 & 2.492242 & 2.93719  & 4.622108 \\
100 & 4.777829 & 2.229433 & 1.717099 & 1.318972 & 1.273402 & 1.306101 & 1.71034  & 2.109493 & 3.473043 \\
200 & 2.989058 & 1.466918 & 1.240388 & 0.952299 & 0.883723 & 0.925386 & 1.183279 & 1.448877 & 2.587599 \\
\hline
\end{tabular}
\end{table}


\begin{table}[H]
\centering
\caption{ $\frac{\text{RMSE}^{NP}_n(\tau)}{\text{RMSE}^{P}_n(\tau)}$ }
\label{tab:rmse_ratio}
\renewcommand{\arraystretch}{1.2}
\begin{tabular}{|c|c|c|c|c|c|c|c|c|c|c|}
\hline
$n \backslash \tau$ & 0.01 & 0.05 & 0.1 & 0.3 & 0.5 & 0.7 & 0.9 & 0.95 & 0.99 \\
\hline
10  & 1.563941 & 1.218972 & 1.422256 & 1.334616 & 1.264677 & 1.24863  & 1.258422 & 1.193925 & 1.56691  \\
50  & 1.735938 & 1.303669 & 1.378897 & 1.256542 & 1.245778 & 1.242178 & 1.311777 & 1.363087 & 1.689322 \\
100 & 2.507701 & 1.449612 & 1.28451  & 1.221025 & 1.255001 & 1.20594  & 1.279303 & 1.349569 & 1.845722 \\
200 & 2.206813 & 1.404069 & 1.292955 & 1.235051 & 1.276296 & 1.247425 & 1.248443 & 1.369565 & 1.891499 \\
\hline
\end{tabular}
\end{table}


As shown in Table~\ref{tab:rmse_p}
and Table~\ref{tab:rmse_np}, the RMSE of the parametric estimator is generally lower than that of the non-parametric estimator. A clearer comparison can be made by calculating the RMSE ratio in Table~\ref{tab:rmse_ratio}, since a lower RMSE indicates a more accurate estimator. Under the assumption of a correctly specified model, the parametric method is more precise.

\begin{table}[H]
\centering
\caption{ $\frac{\text{PLR}^{NP}_n(\tau)}{\text{PLR}^{P}_n(\tau)}$ }
\label{tab:plr_ratio}
\renewcommand{\arraystretch}{1.2}
\begin{tabular}{|c|c|c|c|c|c|c|c|c|c|c|}
\hline
$n \backslash \tau$ & 0.01 & 0.05 & 0.1 & 0.3 & 0.5 & 0.7 & 0.9 & 0.95 & 0.99 \\
\hline
10  & 1.616788 & 1.198526 & 1.455372 & 1.384312 & 1.381471 & 1.392850 & 1.557540 & 1.219940 & 1.931809 \\
50  & 1.706672 & 1.300384 & 1.370660 & 1.268158 & 1.287416 & 1.258040 & 1.383551 & 1.361334 & 1.680849 \\
100 & 2.377747 & 1.424709 & 1.282556 & 1.198635 & 1.260811 & 1.197786 & 1.238279 & 1.391589 & 1.999170 \\
200 & 2.100047 & 1.392118 & 1.275633 & 1.248705 & 1.299327 & 1.204346 & 1.283943 & 1.413369 & 2.001592 \\
\hline
\end{tabular}
\end{table}


Table~\ref{tab:plr_ratio} supports this finding from an economic perspective. The PLR ratios are all greater than 1, which means the non-parametric method results in greater profit loss compared to the parametric one. Therefore, from both statistical and economic perspectives, we can conclude that the parametric method is more accurate.

However, the performance of the parametric estimator is highly dependent on the correctness of the distributional assumption. Unlike the non-parametric estimator, it makes no assumptions about the model and is more robust under uncertainty.


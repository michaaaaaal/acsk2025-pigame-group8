\newpage %Do not remove this command
%%%%%%%%%%%%%%%%%%%% REMOVE THIS WHOLE SECTION AFTER READING IT %%%%%%%%%%%%%%%%%%%% 
\section*{References - Delete After Reading This Section}
\noindent This is where your references will be placed. In this template, you will have two options. Option 1 will require you to fill in the acsk.bib file yourself. Then, when you start citing (using the identifier), it will appear in your reference list. Option two is less organised, but can be quicker. You have to arrange the order and APA-style yourself. Remove the option that you are not going to use.\\

\noindent Notice how you can easily refer to an article \citep{Zako94A}. Place your articles together in the file {\tt bib/acsk.bib}. Another way to refer in the text to, for example, an
article of \citet{Engl82A} or to a book \citep{Tayl86A}. Please have a look at the differences. Each command will show the reference in a different way. \\


%%%%%%%%%%%%%%%%%%%% REMOVE UNTIL HERE! %%%%%%%%%%%%%%%%%%%% 






%%%%%%%%%%%%%%%%%%%% OPTION 1 %%%%%%%%%%%%%%%%%%%% 

\printbibliography %Do NOT remove this unless you want to create your own bibliography without the .bib file. See below.










%%%%%%%%%%%%%%%%%%%% OPTION 2  %%%%%%%%%%%%%%%%%%%% 
%Remove if not used

\iffalse 
\begin{thebibliography}{9} %If you are using this method, remove option 1. Also, remove the \iffalse and \fi commands
\bibitem{latexcompanion} %The identifier is placed within the curly brackets
Michel Goossens, Frank Mittelbach, and Alexander Samarin. 
\textit{The \LaTeX\ Companion}. 
Addison-Wesley, Reading, Massachusetts, 1993.

\bibitem{einstein} 
Albert Einstein. 
\textit{Zur Elektrodynamik bewegter K{\"o}rper}. (German) 
[\textit{On the electrodynamics of moving bodies}]. 
Annalen der Physik, 322(10):891–921, 1905.

\bibitem{knuthwebsite} 
Knuth: Computers and Typesetting,
\\\texttt{http://www-cs-faculty.stanford.edu/\~{}uno/abcde.html}
\end{thebibliography} 
\fi
